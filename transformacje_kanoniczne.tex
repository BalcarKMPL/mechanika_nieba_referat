\begin{frame}{Transformacje Kanonicznowa}
   Podstawowe pytanie: "Skąd brać transformacje kanoniczne?"

   Jest jeden sposób\dots
\end{frame}

\begin{frame}
    W całości prezentacji $q$, $p$, $Q$ oraz $P$ to (co)wektory z $\mathbb{R}^n$ i $\mathbb{R}^{n\ast}$. 

    Trajektorie fizyczne to ekstremale funkcjonału:

    \[ (q, p) \mapsto \int_{t_0}^{t_k} p\cdot \dot{q} - H \text{d}t \]

    te trajektorie, po transformacji kanonicznej, muszą być ekstremalami funkcjonału:

    \[ (Q, P) \mapsto \int_{t_0}^{t_k} P\cdot \dot{Q} - K \text{d}t \]
\end{frame}

\begin{frame}
    Oba funkcjonały są równoważne w szczególności jeżeli ich formy różnią się o pochodną jakiejś funkcji $F$ po czasie:

    \[ (p\cdot \dot{q} - H) \text{d}t = (P\cdot \dot{Q} - K)\text{d}t + \text{d}F \]

    Ktoś mógłby zapytać, "funkcją czego dokładnie jest $F$?". Poprawnych odpowiedzi jest wiele, najczęstsze przypadki to funkcje od $(q, Q, t)$, $(q, P, t)$, $(p, Q, t)$ oraz $(p, P, t)$.

\end{frame}

\begin{frame}
    Dla przykładu, niech $F=F(p, Q, t)$. Wówczas (dodając cichaczem dodatkowy term, który nic nie psuje \emph{pinky promise}) dostaniemy:

    \[ \text{d}F = p \cdot \text{d}q - P \cdot \text{d} Q + (K-H)\text{d}t \]

    Stąd dostajemy następujące zależności:

    \begin{itemize}
        \item $p = \frac{\partial F}{\partial q}$,
        \item $P = -\frac{\partial F}{\partial Q}$,
        \item $K = H + \frac{\partial F}{\partial t}$.
    \end{itemize}

\end{frame}

\begin{frame}
    Przykład:

    Rozważmy tradycyjnie układ oscylatora harmonicznego:

    \[ H = \frac{1}{2}p^2 + \frac{1}{2} q^2 \]

    i funkcję tworzącą:

    \[ F(p,Q) = \frac{1}{2} p^2 \cot Q \]

    wówczas dostajemy zależności:

    \[q=-p \cot Q \quad\quad\quad P=\frac{1}{2} p^2 \frac{1}{\sin^2 Q}\]

    które możemy uprościć do...

\end{frame}

\begin{frame}
    \[ \frac{1}{2}p^2 = P \sin^2 Q \quad\quad \frac{1}{2}q^2 = \frac{1}{2} p^2 \frac{\cos^2 Q}{\sin^2 Q}=P \cos^2 Q \]

    I wsadzając te termy do Hamiltonianu otrzymujemy:

    \[ K(Q, P, t) = P \]

    I dostajemy nasze klasyczne zmienne kąt-działanie.
\end{frame}

\begin{frame}
    Transformacje kanoniczne pochodzące z funkcji tworzących mogą posłużyć nam do drastycznego uproszczenia dynamiki. W szczególności można znaleźć taką transformację, że:

    \[ K(Q, P, t) = 0 \]

    co natychmiast sprawia, że wszystkie trajektorie są stacjonarne. Mowa tutaj o tzw. równaniu Hamiltona-Jacobiego.
\end{frame}

\begin{frame}
    W wielkim skrócie, szukamy funkcji $F(q, P, t)$ takiej, że Kamiltonian będzie się wyrażał zależnością $K=H+\frac{\partial F}{\partial t}$. Jak łatwo policzyć, nowe i stare pędy wyrażają się pochodnymi $F$:

    \[ p=\frac{\partial F}{\partial q} \quad\quad Q=\frac{\partial F}{\partial P} \]

    Pierwsze $n$ równań, wraz z równaniem H-J $0=H+\frac{\partial F}{\partial t}$, zadaje układ $n+1$ równań różniczkowych cząstkowych, gdzie $Q$ są parametrami, a nie prawdziwymi zmiennymi funkcji $F$. Dla przykładu wykorzystamy oscylator harmoniczny.
\end{frame}

\begin{frame}
    Zaczynamy od $H=\frac{1}{2}(p^2+q^2)$. Zakładamy, że funkcję $F$ można zapisać jako sumę rozdzielonych zmiennych:

    \[ F(q,P,t) = F_t(t,P) + F_q(q,P) \]

    Używając równania $\partial_t F_t = -H$ i zauważając, że $H$ jest stałe, zatem równe jedynemu z parametrów, jaki memy do dyspozycji, dostajemy:

    \[ F_t = -P t \]

    Analogicznie rozwiązując dla $F_q$ dostajemy:

    \[ F_q = \int_0^q \sqrt{2 P - \overline{q}^2} \text{d} \overline{q} \]

    i natychmiastowo:

    \[ F = -P t + \int_0^q \sqrt{2 P - \overline{q}^2} \text{d} \overline{q} \]
\end{frame}

\begin{frame}
    Licząc $Q = \partial_P F$ dostajemy:

    \[ Q = -t +\arcsin\left(\frac{q}{\sqrt{2 P}}\right) \]

    i po algebraicznych przekształceniach:

    \[ q = \sqrt{2 P} \sin\left(Q + t\right) \]
    \[ p = \sqrt{2 P - q^2} = \sqrt{2 P} \cos\left(Q + t\right) \]

    W ten sposób rozwiązaliśmy zagadnienie oscylatora harmonicznego jedną z najbardziej skomplikowanych metod, 
\end{frame}

\begin{frame}
    Weźmy trudniejszy przykład, cząstkę w $\mathbb{R}^3$ w potencjale centralnym:

    \[ H = \frac{1}{2m}\left(p_r^2 + \frac{1}{r^2}p_\theta^2 + \frac{1}{r^2 \sin^2\theta}p_\theta^2\right)+U(r) \]

    Zakładając rozdzielone zmienne:

    \[ F = R(r, P) + \Theta(\theta, P + \Phi(\phi, P) + T(t, P) \]

    Zatem równanie Hamiltona Jacobiego daje nam:

    \[ T' + \frac{1}{2m}\left(R'^2 + \frac{1}{r^2}\Theta'^2 + \frac{1}{r^2 \sin^2\theta} \Phi'^2 \right) + V(r) = 0 \]
\end{frame}

\begin{frame}
    \[ \frac{1}{2m}\left(R'^2 + \frac{1}{r^2}\Theta'^2 + \frac{1}{r^2 \sin^2\theta} \Phi'^2 \right) + V(r) = -T' \]

    Jak widać, term po prawej zależy wyłącznie od $t$, zaś term po lewej od $t$ nie zależy, zatem obie strony równania są stałe, co daje $T(t,P) = -t E$ (gdzie $E$ to jeden z nowych "pędów") oraz:

    \[ \frac{1}{2m}\left(R'^2 + \frac{1}{r^2}\Theta'^2 + \frac{1}{r^2 \sin^2\theta} \Phi'^2 \right) + V(r) = E \]
\end{frame}

\begin{frame}
    Upraszczając poprzednie równanie:

     \[ r^2 \sin^2\theta \cdot R'^2 + \sin^2\theta \cdot \Theta'^2 + r^2 \sin^2\theta \cdot 2m(V(r) - E) = -\Phi'^2 \]

     Znów, część po prawej zależy wyłącznie od $\phi$, a część po lewej nie, zatem dostajemy $\Phi = L_z \phi$ (gdzie $L_z$ to kolejny z nowych "pędów") oraz:

    \[ r^2  \cdot R'^2 + r^2  \cdot 2m(V(r) - E) = -\Theta'^2 - \frac{L_z^2}{\sin^2\theta}\]

    Rozwiązując analogicznie jak wcześniej dostajemy:

    \[ R = \sqrt{2m}\int\sqrt{E-U(r)-\frac{L^2}{2mr^2}}\,\text{d}r  \quad\quad \Theta = \int \sqrt{L^2 - \frac{L_z^2}{\sin^2\theta}}\,\text{d}\theta \]
     
\end{frame}

\begin{frame}
    Oczywiście, największą trudnością jest rozwiązanie równania:

    \[ H(q, \frac{\partial F}{\partial q}, t) + \frac{\partial F}{\partial t} = 0 \]

    Należy się zastanowić, czy taka funkcja istnieje zawsze (tj. dla każdego Hamiltonianu).

\end{frame}

\begin{frame}
    Wiadomo z analizy matematycznej, że:

    \[ \dot{F}=\frac{\partial F}{\partial q} \dot{q} + \frac{\partial F}{\partial P} \dot{P} + \frac{\partial F}{\partial t} \]

    a z poprzednich slajdów, że:

    \[ \frac{\partial F}{\partial q} = p \quad\quad \frac{\partial F}{\partial P} = 0 \quad\quad \frac{\partial F}{\partial t} = -H \]

    zatem mamy, że:

    \[ \dot{F} = p \cdot \dot{q} - H = \mathcal{L} \]

    całkując po czasie dostajemy 

    \[ F = \int_{t_0}^{t_k} \mathcal{L}\text{d} t = S \]

    Czyli ta funkcja tworząca to po prostu działanie.
\end{frame}

\begin{frame}
    Referencje:

    \begin{itemize}
        \item \url{https://astro.pas.rochester.edu/~aquillen/phy411/lecture2.pdf} - można tam znaleźć przyjemne interpretacje geometryczne
        
        \item \url{https://phys.libretexts.org/Bookshelves/Classical_Mechanics/Variational_Principles_in_Classical_Mechanics_(Cline)/15\%3A_Advanced_Hamiltonian_Mechanics} - zawiera m.in. przykłady HJE dla większych układów niż oscylator harmoniczny
    \end{itemize}
\end{frame}

